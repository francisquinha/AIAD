% 	
%	
%
\documentclass[a4paper, 12pt, twoside]{scrreprt}

% Encoding, islands characters.
%\usepackage{ucs}
\usepackage[utf8]{inputenc}
\usepackage[portuguese]{babel}
\usepackage{t1enc}
%\usepackage{csquotes} % Problemas com aspas na biblioteca.

\usepackage[intoc]{nomencl}
\usepackage{enumerate,color}
\usepackage{url}
\usepackage[pdfborder={0 0 0}]{hyperref}
\usepackage{appendix}
\usepackage{eso-pic}
\usepackage{amsmath}
\usepackage{amssymb}
\usepackage[nottoc]{tocbibind}



\usepackage[format=plain,labelformat=simple,labelsep=colon]{caption}
\usepackage{placeins}

% comprimidos
\usepackage{tabularx}

% Desenhos e fotos.
\usepackage{graphicx}
\usepackage[sf,normalsize]{subfigure}
\usepackage{tikz}
\usetikzlibrary{calc}
\usepackage{circuitikz}
\usepackage{tikz,pgffor}
\usetikzlibrary{calc,through,intersections, matrix}
\usepackage{tkz-euclide}
\usetkzobj{all}
\usepackage{geometry}

% units package
\usepackage{siunitx}
\usepackage{cancel}

% Suplemento Bibliográfico.
\usepackage[style=ieee, backend=biber]{biblatex} 
\addbibresource{heimildaskra.bib} % Fontes são baixadas a partir desta lista.

% Configurations
%\graphicspath{{figs/}}

%\usepackage{fancyhdr}
 
%\pagestyle{fancy}
%\fancyhf{}
%\rhead{Share\LaTeX}
%\lhead{Guides and tutorials}
%\rfoot{Page \thepage}


\setlength{\parskip}{\baselineskip}
\setlength{\parindent}{0cm}
\raggedbottom
% \setkomafont{subsection}{\normalfont\sffamily}

% Como modificação de fios para ser
% \setkomafont{captionlabel}{\itshape}
% \setkomafont{caption}{\itshape}

% Mais bela solução << esta é uma escolha vergonhosa
 \setkomafont{captionlabel}{\itshape}
 \setkomafont{caption}{\itshape}
 \setkomafont{section}{\FloatBarrier\Large}
 \setcapwidth[l]{\textwidth}
%\setcapindent{1em}


% Times new roman
%\usepackage[T1]{fontenc}
%\usepackage{mathptmx}

%
%	Preencha Informações sobre o titulo, departamento e autor.
%
\def\thesisyear{2016}       				% Ano Projecto Entrega
\def\thesismonth{Novembro}					% Mês do Projecto final
\def\theisnrofauthors{3}
\def\thesisauthors{Marina Camilo - up201307722 - up201307722@fe.up.pt
				\\ Diogo Ferreira - up201502853 - diogoff@fe.up.pt
                \\ Ângela Cardoso - up200204375 - angela.cardoso@fe.up.pt}						% Authors
\def\thesistitle{T05: Exploração de Marte} 		% Titulo
\def\thesisshorttitle{Agentes e Inteligência Artificial Distribuida} 	% Tiulo Curto

\def\thesisnroftutors{2}						% 1 = singular, >1 plural
\def\thesistutors{Eugénio da Costa Oliveira \\
				  Henrique Daniel de Avelar Lopes Cardoso }	% Formador

\begin{document}

\begin{titlepage}
\thispagestyle{empty}

\begin{tikzpicture}[remember picture,overlay]
  \coordinate [above=0cm] (toppoint) at (current page.north);
  % Logo Universidade.
  \node[anchor=north] (logonator) at ($(toppoint)+(0cm,-4.2cm)$) {


\includegraphics[width=8cm]{./FEUP_Logo.png}
    };

  % Titulo
  \node[anchor=north, align=center] (titlenator) at ($(logonator.south)+(0cm,-2.0cm)$) {
    \huge \sffamily \bfseries \thesistitle
    };

  \coordinate [above=0cm] (bottompoint) at (current page.south);

  % Promoçao de texto localização e atribuições
  \node[anchor=base, 
  		minimum width=\paperwidth, 
        align=center] at ($(bottompoint)+(0cm,5cm)$)  
    {
    \parbox{
      1\paperwidth
      }{
      \centering
      \bfseries\sffamily \Large 
	  4ºAno - Sistemas Distribuidos \\
	  \thesismonth~\thesisyear
      }
    };

  % nome do autor
  \node[anchor=north, align=center] (authornator) at ($(titlenator.south)+(0cm,-1.0cm)$) {
  	\parbox{
      1\paperwidth
      }{
      \centering
      \normalfont \Large \sffamily \thesisauthors
      }
    };
    
    %\normalfont \Large \sffamily \thesisauthors
	%};
    
  % nomes dos Supervisores
  \node[anchor=north, align=center] (advisornator) at ($(authornator.south)+(0cm,-1.0cm)$) {
	\parbox{
      1\paperwidth
      }{
      \centering
      \sffamily \Large Professores\\
      \thesistutors
      }
    };
    
\end{tikzpicture}

\end{titlepage}
\pagenumbering{arabic}

\setcounter{page}{2}
\newpage

\tableofcontents
%\listoffigures
%\listoftables

\clearpage
\chapter{Enunciado}
\pagenumbering{arabic}
\setcounter{page}{3}

\section{Descrição do cenário}
No âmbito da unidade Curricular de Agentes e Inteligência Artificial Distribuída o grupo propôs-se a implementar um Sistema Multi-Agente para simulação de um cenário de extração de minérios em Marte.
Sendo assim, é necessário um conjunto de agentes com a tarefa de explorar o planeta Marte em busca de minérios, e de transportar a maior quantidade possível para a base. Para tal, existem três tipos de Agentes:

\begin{itemize}
	\item {\textbackslash}\textit{Spotter}– Procura fontes de minérios e inspeciona-los para determinar se podem ser explorados. 
	\item {\textbackslash}\textit{Producer}– É chamado a uma fonte de minério por um \textit{Spotter} para extrair o máximo de minério possível nessa fonte. 
	\item {\textbackslash}\textit{Transporter}– É alocado pelo \textit{Producer} para carregar o minério obtido para a base.
\end{itemize}

De forma a facilitar a procura, todos os agentes podem localizar fontes de minérios e enviar a sua localização para os \textit{Spotter} que os analisarão. A escolha do \textit{Producer} por parte do \textit{Spotter} segue um protocolo de negociação. A alocação dos \textit{Transporters} a uma determinada fonte segue também um protocolo de negociação, iniciado pelo \textit{Producer}. Esta alocação, terá em conta a quantidade de minério a transportar, de modo a determinar mais corretamente o número necessário de \textit{Transporters}.

\section{Objectivos do trabalho}
Um dos objetivos deste trabalho é implementar os agentes de forma a que a simulação da exploração do cenário de Marte se torne o mail eficiente possível. 
No caso do \textit{Spotter} será implementado um algoritmo que dividirá a área explorada pelos \textit{Spotter} existentes. Será também implementado um protocolo de negociação que irá determinar que \textit{Producer} será melhor para se deslocar para o local do minério encontrado.
No caso do \textit{Producer} será implementado um protocolo de negociação que irá determinar que ou quais \textit{Transporters} serão mais eficientes a recolher o minério.

\section{Resultados esperados e forma de avaliação}
Inicialmente serão implementadas apenas as funcionalidades básicas de cada Agente como tal:
A 1º fase de avaliação será verificar o sucesso da implementação do comportamento de cada agente. Após se garantir que todos os agentes realizam o seu papel corretamente passamos para a fase seguinte, a fase de implementação de restrições. 
Nesta 2º fase, irá avaliar–se se os \textit{Transporters} chamados não ultrapassam a sua capacidade, se o \textit{Transporters} chamados conseguem recolher todo o minério presente. 
Após estas fases, implementaremos algoritmos de forma a tornar mais eficiente esta demanda, avaliando se as alocações dos demais agentes correspondem ao mais disponível na altura. Se o mapa fica corretamente dividido entre os \textit{Spotters} e se o tempo de simulação foi o mínimo para o caso em questão.

 
\chapter{Plataforma/Ferramenta}
%escrever aqui
\section{Para que serve}
%escrever aqui
\section{Descrição das características principais}
%escrever aqui
\section{Realce das funcionalidades relevantes para o trabalho}
%escrever aqui

\chapter{Especificação}
%escrever aqui
\section{Identificação e caracterização dos agentes (arquitectura, comportamento, estratégias)}
%escrever aqui
\section{Protocolos de interacção}
%escrever aqui
\section{Faseamento do projecto}

\begin{table}[htb]
\centering
\caption{Fases previstas para o projecto}
    \sffamily \begin{tabularx}{1.0\textwidth}{ p{3cm}  p{10,5cm} }
    \hline
    \textbf{1º Ponto} \hfill & Construir ambiente de simulação na tecnologia \textit{Repast} \\ \hline
    \textbf{2º Ponto} \hfill & Criação do \textit{Spotter} com as função de explorar e dividir território a explorar. \\ \hline
    \textbf{3º Ponto} \hfill & Criação do \textit{Producer} com a função básica de produzir. Melhoramento do \textit{Spotter} para chamar \textit{Producers}. \\ \hline
    \textbf{4º Ponto} \hfill & Criação do \textit{Transporter} sem limite de capacidade e apenas com a função básica de transportar. Melhoramento do \textit{Producer} para chamar \textit{Transporters}. \\ \hline
    \textbf{5º Ponto} \hfill & Melhoria dos Agentes \textit{Spotter}, \textit{Producer} e \textit{Transporter}. \\ \hline
    \textbf{6º Ponto} \hfill & Defenir estratégias de forma a tornar a exploração de Marte o mais eficiente possível. \\ \hline
    \end{tabularx} \normalfont
\label{table:Emissivity}
\end{table}

\chapter{Recursos}
%escrever aqui
\section{Bibliografia}
%escrever aqui
\section{Software}
%escrever aqui


\chapter{Exemplo Figura e tabela}

\begin{figure}[!htb]
\centering
\includegraphics[width=0.60\textwidth]{FEUP_Logo}
\caption[alguma coisa.]{Mais aluguma coisa} \label{fig:Array}
\end{figure}

\begin{table}[htb]
\centering
\caption{Alguma coisa.}
     \sffamily \begin{tabularx}{1.0\textwidth}{ p{5cm}  p{5cm}  p{5cm} }
    \hline
   \textbf{nome1} \hfill & \textbf{nome2} \hfill & \textbf{nome3} \\ \hline
    1.1 & 1.2 & 1.3\\
    2.1 & 2.2 & 2.3\\ \hline
    \end{tabularx} \normalfont
\label{table:Emissivity}
\end{table}

\section{Baixando a partir de outro documento}
Exemplos de autorizado~\cite{dirac} e tambem~\cite{einstein}.
\input{YtraEfniFyrirInclude}


%
% Fontes
%
\printbibliography[heading=bibintoc] % Adiciona fontes e traz uma tabela de conteudos
%escrever aqui


%
% Anexos
%
\appendix
\renewcommand{\chaptername}{Anexos}
\chapter{Anexos}
Dicas úteis e waypoints

\end{document}